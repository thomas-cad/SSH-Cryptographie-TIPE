\chapter{Bilan Projet}

Tout d'abord, le choix de ce sujet n'a pas été une évidence, nous avons choisi d'orienter notre TIPE sur la cryptographie, mais sans réelle idée du domaine d'application. Pour nous aider, nous nous sommes documentés sur les grands principes de la cryptographie, ceci a débouché sur l'étude de différents domaines. Comme les terminaux de paiement ou le cryptage et l'authentification de mails ou encore les signatures électroniques. C'est l'utilisation de GitHub qui nous a orienté vers ce choix de sujet en désirant utiliser le protocole SSH pour connecter notre ordinateur à GitHub.\\

Pour réaliser notre projet, nous avons mis en place un environnement numérique pour le travail collaboratif. Tout d'abord, par le déploiement d'une équipe Teams qui nous a permise de centraliser nos visio-conférences, nos communications ainsi que nos outils de travail. Ensuite, nous avons utilisé OneNote pour centraliser nos recherches ainsi que les notes lors de nos réunions et points avec notre tuteur. Miro a été également un outil important pour créer des cartes mentales pour découvrir les sujets lors de nos recherches. Pour planifier les différentes étapes du projet et suivre sa progression. Enfin, nous avons utilisé l'outil de la suite Office intégré à notre Teams, planner.\\

Ce rapport a été rédigé à l'aide de \LaTeX{}, nous souhaitions découvrir cet outil pour la rédaction d'articles scientifiques de qualité. Nous avons pour cela créé notre projet sur Overleaf pour pouvoir collaborer. Nous avons pu découvrir de nombreuses fonctionnalités offertes par \LaTeX{} avec des nombreux packages à notre disposition. La gestion des sources de nos recherches a été réalisée avec Zotero pour obtenir un meilleur suivi et faciliter la réalisation de notre bibliographie avec l'export d'un fichier {\ttfamily .bib}. Ces compétences acquises nous seront utiles pour réaliser nos futurs documents scientifiques.\\

Ce projet a été l'occasion d'utiliser et d'approfondir des notions vues au sein du CITISE. Ceci a notamment été le cas avec l'étude de trames réseaux qui nous a permis d'approfondir certaines notions vues en cours de réseaux. L'étude de ce protocole nous permet également de mieux comprendre l'utilisation que nous en faisons en AT33 (R32 InfoSpé) pour la communication entre notre Raspberry et notre ordinateur. Pour conclure, ce projet nous a permis de découvrir le domaine de la cryptographie, de renforcer nos connaissances, tant au niveau scientifique que pour la gestion de projet, sans oublier l'étude de la sécurité informatique avec une approche technique.